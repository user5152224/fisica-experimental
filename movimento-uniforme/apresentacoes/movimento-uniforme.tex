%!latex
\documentclass[
  leqno,
  aspectratio=169
]{beamer}

% O pacote amsmath é carregado automaticamente pela classe BEAMER
\usepackage{
  amsmath,
  dsfont,
  enumitem,
  helvet,
  hyperref,
  lipsum,
  mathrsfs,
  ragged2e,
  tikz,
  xcolor,
}

\usepackage[brazil]{babel}

\usetikzlibrary{
  arrows.meta,
  calc,
}

% Controlar o estilo do Beamer   
\beamertemplatenavigationsymbolsempty 
\newcommand*{\theorembreak}{\usebeamertemplate{theorem end}\framebreak\usebeamertemplate{theorem begin}}
\setbeamercolor{section number projected}{bg=structure,fg=white}
\setbeamercolor{bibliography entry author}{fg=black}
\setbeamercolor{structure}{fg=green!50!black}
\setbeamerfont{frametitle}{shape=\slshape, family=\sffamily}  
%\setbeamerfont{section number projected}{size=\normalsize}
%\setbeamerfont{title}{shape=\itshape, family=\rmfamily}
\setbeamertemplate{bibliography item}{\insertbiblabel}
\setbeamertemplate{section in toc}[sections numbered]
\setbeamerfont{section in toc}{shape=\slshape, family=\sffamily}
\setbeamertemplate{theorems}[numbered]
\apptocmd{\frame}{}{\justifying}{} %Justificar todo o texto no frame

\newtheorem{axi}{Axioma}
\newtheorem{cor}{Corol\'ario}
\newtheorem{Def}{Defini\c c\~ao}
\newtheorem{exe}{Exemplo}
\newtheorem{lem}{Lema}
\newtheorem{obs}{Observa\c c\~ao}
\newtheorem{pro}{Proposi\c c\~ao}
\newtheorem{teo}{Teorema}

% Fazer o título da apresentação.
\newcommand{\titulo}[3]{
  \begingroup    
    \setbeamertemplate{headline}{}
    \begin{frame}[noframenumbering]    
      \centering     
      \footnotesize
      \parbox[c]{6cm}{  
        {% 
          UNIVERSIDADE\hfill FEDERAL\hfill DA\hfill PARA\'IBA%
        }
        \vskip 1pt
        \fontsize{7}{7}\selectfont
        CENTRO \hfill DE\hfill CI\^ENCIAS\hfill EXATAS\hfill E\hfill DA NATUREZA%
        \par
      }      
      \vskip 1.4cm
      {
        \usebeamercolor[fg]{structure}
        \large\bfseries\sffamily\slshape\MakeUppercase{%
          #1
        }
      }
      \vskip .42cm
      {%
        \itshape por
      }
      \vskip .42cm
      {%
        \sffamily #2
      }
      \vskip .2cm
      e
      {%
        \sffamily #3
      }
      \tiny\par
      {%
        \tt\href{mailto:harllen.araujo@gmail.com}{<harllen.araujo@gmail.com>}
      }
      \vskip .63cm
      Jo\~ao Pessoa\par\the\year
    \end{frame}
  \endgroup
}

% Barra de progresso adaptado de: https://tex.stackexchange.com/questions/59742/progress-bar-for-latex-beamer.
\makeatletter
  \def\progressbar@progressbar{}  % the progress bar
  \newcount\progressbar@tmpcounta % auxiliary counter
  \newcount\progressbar@tmpcountb % auxiliary counter
  \newdimen\progressbar@pbht      % progressbar height
  \newdimen\progressbar@pbwd      % progressbar width
  \newdimen\progressbar@rcircle   % radius for the circle
  \newdimen\progressbar@tmpdim    % auxiliary dimension
  
  \progressbar@pbwd=\linewidth
  \progressbar@pbht=1pt
  
  % A barra de progresso
  \def\progressbar@progressbar{%
    \progressbar@tmpcounta=\insertframenumber
    \progressbar@tmpcountb=\inserttotalframenumber
       \progressbar@tmpdim=\progressbar@pbwd
    \multiply\progressbar@tmpdim by \progressbar@tmpcounta
    \divide\progressbar@tmpdim by \progressbar@tmpcountb
  
    \begin{tikzpicture}
      \draw [ white!85!black,
              line width=\progressbar@pbht,
              line cap=round,
            ] (0pt, 0pt) -- ++ (\progressbar@pbwd,0pt) node [ text=green!50!black, 
                                                              midway,
                                                              below=2pt
                                                            ] {
                                                                %\fontsize{9.1}{9.1}\selectfont
                                                                \slshape\bfseries
                                                                \MakeUppercase{\@currentlabelname}
                                                              };
      \draw [ green!50!black,
              line width=\progressbar@pbht,
              line cap=round,
            ] (0pt, 0pt) -- ++ (\the\dimexpr\progressbar@tmpdim-\progressbar@rcircle\relax,0pt);
      \node[ %draw=green!50!black,
             text width=3.5em,
             align=center,
             inner sep=1pt,
             text=green!50!black,
             anchor=west,
             rounded corners=.5pt
           ] at (\progressbar@pbwd,0) {%
                                        \fontsize{7.7}{7.7}\selectfont\insertframenumber/\inserttotalframenumber
                                      };
    \end{tikzpicture}%
  }
  \addtobeamertemplate{headline}{}
  {%
    \begin{beamercolorbox}[ wd=\paperwidth,
                            ht=7ex,
                            center,
                            dp=1ex
                          ]{white}%
      \progressbar@progressbar%
    \end{beamercolorbox}%
  }
\makeatother

% Define um estilo de teorema
\newtheoremstyle{slanted}
{10pt}
{10pt}
{\slshape}
{}
{\bfseries}
{}
{0.5em}
{\thmname{\bfseries #1} \thmnumber{\scshape #2}\thmnote{ \scshape(#3)}\quad}

\theoremstyle{slanted}

% Referências cruzadas.
\def\rf#1{
  {
    \usebeamercolor[fg]{structure}\ff\ref{#1}
  }
}

\def\prova{
  {
    \usebeamercolor[fg]{structure}\itshape Prova.\ %
  }
}
\def\AEs{{\bf AEs}}
\def\AEx{{\bf AEx}}
\def\AS{{\bf AS}\ }
\def\Cap{\mathop{\textstyle\bigcap}}
\def\Cup{\mathop{\textstyle\bigcup}}
\def\C{\mathds{C}}
\def\Join{\mathop{\textstyle\bigcurlyvee}}
\def\K{\mathds{K}}
\def\L{\mathds{L}}
\def\N{\mathds{N}}
\def\Meet{\mathop{\textstyle\bigcurlywedge}}
\def\Q{\mathds{Q}}
\def\R{\mathds{R}}
\def\U{{\cal U}}
\def\Vee{\mathop{\textstyle\bigvee}}
\def\Wedge{\mathop{\textstyle\bigwedge}}
\def\Z{\mathds{Z}}
\def\[{\begin{equation}}
\def\]{\end{equation}}
\def\af{{\it a fortiori}}
\def\a{\wedge}
\def\bf{\bfseries}
\def\cal{\mathcal}
\def\cp{\mathbin{\neg}}
\def\ct#1{{[#1 \citenum{#1}]}}
\def\d{\,d}
\def\fr#1#2{\frac{#1}{#2}}
\def\frak{\mathfrak}
\def\join{\curlyvee}
\def\la{\leftarrow}
\def\lge{\langle}
\def\lla{\longleftarrow}
\def\llra{\longleftrightarrow}
\def\lra{\longrightarrow}
\def\meet{\curlywedge}
\def\ora#1{\mkern3mu\overrightarrow{\mkern-3mu#1\mkern3mu}\mkern-3mu}
\def\o{\vee}
\def\pma{{\bfseries PMA}}
\def\prova{\noindent{\usebeamercolor[fg]{structure}\bfseries Prova}\quad}
\def\qed{\nopagebreak\hbox{ }\hfill{\usebeamercolor[fg]{structure}\rule{.42em}{1em}}\bigskip}
\def\ra{\rightarrow}
\def\res{\noindent{\itshape\bfseries Resolu\c c\~ao.\/}\ }
\def\rf#1#2{{\usebeamercolor[fg]{structure}\bfseries\hyperref[#2]{#1 \ref{#2}}}}
\def\rge{\rangle}
\def\scr#1{\mathscr{#1}}
\def\sen{\,{\rm sen}\,}
\def\tvm{{\bf TVM}}
\def\prll{\mathbin{\hbox{/\kern-1pt/}}}
\def\olra#1{\overleftrightarrow{\mkern-4mu#1}}



\begin{document}
  \titulo{%
    Autovalores, autovetores e diagonaliza\c c\~ao  
  }

  \begin{frame}[noframenumbering, plain]{Sum\'ario}
      \tableofcontents[pausesections]
  \end{frame}
  
  \section{Introdu\c c\~ao}
  \begin{frame}
  \end{frame}
  
  \section{Autovalores e autovetores}
  \begin{frame}{Autovalores e autovetores}
    \begin{Def}
      \justifying
      Sejam $V$ um espa\c co vetorial sobre um corpo $\K$ $($usualmente $\K\in\{\R,\C\}$$)$ e ${T:V\ra V}$ uma transforma\c c\~ao linear. Dizemos que $\lambda\in\K$ \'e um {\bfseries autovalor}, ou valor caracter\'istico, ou ainda valor pr\'oprio se, e somente se, existir um vetor $v\in V$, n\~ao nulo, tal que $Tv=\lambda v$. O vetor associado ao autovalor $\lambda$ \'e chamado de {\bfseries autovetor}, vetor caracter\'istico, ou vetor pr\'oprio.
    \end{Def}
  \end{frame}

  \begin{frame}
    Numa perspectiva geom\'etrica $T$ \'e preserva a reta (espa\c co de unidimensional) que tem $v$ como um de seus vetores diretores, {\it viz.}
    \[
      [v]=\bigr\{u:\exists\varkappa(\varkappa\in\K\,\a\, u=\varkappa v)\bigr\},
    \]
    o subespa\c co vetorial gerado por $v$.
  \end{frame}

  \begin{frame}
    \begin{exe}
      Considere a aplica\c c\~ao $H_\lambda:V\ra V$, dada por $H_\lambda=\lambda I$. Supondo que $V\neq\{0\}$, segue-se que $\lambda$ \'e um autovalor de $H_\lambda$.
    \end{exe}
  \end{frame}

  \begin{frame}
    \begin{exe}
      \justifying
      Sejam $\scr D$ o cojunto das fun\c c\~oes diferenci\'aveis em $A$, aberto de $\R$, com imagens em $\R$, e $D$ o operador derivada. \'E conhecido do c\'alculo que $D$ \'e linear em $\scr D$ e para todo $\lambda\in\R$, tem-se
      \[
        De^{\lambda x}=\lambda e^{\lambda x}.
      \]
      Em palavras a fun\c c\~ao $f\in\scr D$ definida por $f(x)=e^{\lambda x}$ \'e um autovalor de $D$ com autovetor $f$.
    \end{exe}
  \end{frame}

  \begin{frame}
    \begin{exe}
      Seja $T:\R^2\ra \R^2$ cuja matriz relativa \`a base can\^onica do $\R^2$ \'e
      \[
        \begin{bmatrix}
          2  & 3 \cr
          -1 & 1 \cr
        \end{bmatrix}
      \]
    \end{exe}
    Seus autovalores s\~ao dados resolvendo-se o sistema
    \[
      \left\{
      \begin{aligned}
        2x+3y &=\lambda x\cr
        -x+y  &=\lambda y\cr
      \end{aligned}
      \right.
    \]
  \end{frame}

  \begin{frame}
    Determinamos
    \[
      (1-\lambda)y=x,
      \label{eq1901261618}
    \]
    Como queremos um vetor n\~ao nulo, segue-se de (\ref{eq1901261618}) que $x,y\neq 0$ em consequ\^encia
    \[
      \bigl((2-\lambda)(1-\lambda)+3\bigr)y=0.
    \]
  \end{frame}

  \begin{frame}
    acarreta
    \[
      \lambda^2-3\lambda+5=0
    \]
    como
    \[
      \Delta=9-4\cdot1\cdot5=-11<0
    \]
    n\~ao existe autovalores.
  \end{frame}
  
  \begin{frame}
    \begin{exe}[Rota\c c\~oes]
      \justifying
      Podemos naturalmente considerar $\C^2$ como um espa\c co vetorial sobre $\R$. Definamos a transforma\c c\~ao linear $R_\theta:\C^2\ra \C^2$, dada por $R_\theta z=e^{i\theta}z$, geometricamente multiplicar por um complexo significa rotacionar por um certo \^angulo e aplicar uma homotetia.
    \end{exe}
  \end{frame}
  
  \begin{frame}
    Supondo $\theta\in R\cp \pi\Z$, ent\~ao n\~ao existe $\lambda\in\R$ tal que
    \[
      \lambda z=R_\theta z=e^{i\theta}z.
      \label{eq1001251946}
    \]
    Com efeito, (\ref{eq1001251946}) nos diz que 
    \[
      \lambda=e^{i\theta}=\cos\theta+i\sin\theta,
    \]
    o que por sua vez acarreta que $\sin\theta=0$, o que n\~ao \'e o caso, pois escolhemos $\theta\in R\cp \pi\Z$. Logo, $R_\theta$, n\~ao possui autovalores.

  \end{frame}
  
  \begin{frame}
    \begin{exe}
      Seja $V$ o espa\c co vetorial de todas as fun\c c\~oes cont\'inuas de $R$ em $R$. Seja $T$ o operador linear em $V$ definido por
      \[
        (Tf)(x)=\int_0^xf(t)\, dt.
      \]
      Por {\it reductio ad absurdum} seja $f\in V\cp\{0\}$ tal que $Tf=\lambda f$. Do c\'alculo, segue-se que existe apenas uma \'unica fun\c c\~ao tal que
      \[
        \int_0^xf(t)\,dt=\lambda f(x),
      \]
    \end{exe}
  \end{frame}

  \begin{frame}
    a saber $f(x)=e^{\lambda x}$. No entanto, de
    \[
      \int_0^xf(t)\,dt=\lambda f(x),
    \]
    e do teorema fundamental do c\'alculo tem-se
    \[
      \lambda e^{\lambda x}=\int_0^xe^{\lambda x}\,\d x=e^{\lambda x}-1,
    \]
    donde
    \[
      e^{\lambda x}={1\over 1-\lambda}.
    \]
    o que \'e um absurdo.
  \end{frame}
  
  \begin{frame}
    \begin{Def}
      Dada uma transforma\c c\~ao linear $T$ em um espa\c co vetorial sobre um corpo $\K$, sabe-se que
      \[
        A_\lambda=\ker(T-\lambda I)=\bigl\{v:v\in V\,\a\,(T-\lambda)v=0\bigr\},
      \]
      \'e um espa\c co vetorial. Nos referiremos a $A_\lambda$ como o {\bfseries autoespa\c co} associado ao autovalor $\lambda$.
    \end{Def}
  \end{frame}
  
  \begin{frame}
    Exemplos com poucas dimens\~oes podem ser tratados via sistemas e an\'alise de casos, mas, e se a matriz da transforma\c c\~ao tiver milh\~oes ou at\'e bilh\~oes de entradas? 
  \end{frame}

  \begin{frame}
    Qual a raz\~ao de se considerar uma matriz com um n\'umero muito grande de entradas?
  \end{frame}

  \begin{frame}{O porqu\^e de se considerar matrizes gigantes}
    Cito alguns ramos da ci\^encia onde faz-se a necessidade de an\'alise do conjunto de autovalores de matrizes enormes:
    \begin{enumerate}[label = \Roman*.]
      \item{%
        Algoritmos de ranking;
      }
      \item{%
        Din\^amica estrutural e engenharia civil;
      }
      \item{%
        Mec\^anica qu\^antica e qu\'imica computacional;
      }
      \item{%
        Reconhecimento facial e compress\~ao (PCA);
      }
      \item{%
        Redes sociais e an\'alise de grafos.
      }

    \end{enumerate}

  \end{frame}

  \section{Polin\^omio caracter\'istico}
  \begin{frame}{Polin\^omio caracter\'istico}
    A solu\c c\~ao engenhosa encontra-se na \'algebra. Ela jaz nas ra\'izes de um polin\^omio especial, a saber, o {\bf polin\^omio caracter\'istico}.
  \end{frame}

  \begin{frame}
    Seja $V$ um espa\c co vetorial n\~ao trivial e $T:V\ra V$ uma transforma\c c\~ao linear. O {\bf polin\^omio caracter\'istico} $p_T$ \'e definido como
    \[
      p_T(\lambda)=\det(\lambda-T)\qquad\text{ou}\qquad p_T(\lambda)=\det(T-\lambda),
    \]
    pois o que nos interessa aqui s\~ao as suas ra\'izes.
  \end{frame}

  \begin{frame}{Indiferen\c ca do polin\^omio caracter\'istico \`a base}
    O polin\^omio caracter\'istico \'e determinado via uma representa\c c\~ao matricial dum operador numa base ordenada $\frak a\in\K^n$, com $n\in\omega$. O que acontece se eu representar o operador por outra base $\frak b$?
  \end{frame}

  \begin{frame}{Indiferen\c ca do polin\^omio caracter\'istico \`a base}
    Sabe-se que se $[T]_{\frak c}$ \'e a representa\c c\~ao matricial de $T$ na base ${\frak c}\in\{\frak a,\frak c\}$, ent\~ao vale a identidade
    \[
      [T]_{\frak b}=[I]_{\frak b}^{\frak a}\,[T]_{\frak a}\,[I]_{\frak a}^{\frak b}.
    \]
  \end{frame}

  \begin{frame}{Indiferen\c ca do polin\^omio caracter\'istico \`a base}
    Sabendo que $P=[I]_{\frak b}^{\frak a}=([I]_{\frak a}^{\frak b})^{-1}$, segue-se que $[T]_{\frak a}$ e $[T]_{\frak b}$, s\~ao similares. Consequentemente,
    \[
      \begin{aligned}
        \det(\lambda-[T]_\frak b)&=\det\bigl(P(\lambda-[T]_\frak a)P^{-1}\bigr)\cr
                                 &=\det(P)\det\bigl((\lambda-[T]_\frak a)\bigr)\det(P^{-1})\cr
                                 &=\det(P)\det\bigl((\lambda-[T]_\frak a)\bigr)\det(P)^{-1}\cr
                                 &=\det\bigl(\lambda-[T]_\frak a\bigr).\cr
      \end{aligned}
    \]
    Portanto, o polin\^omio caracter\'istico \'e inexor\'avel \`a mudan\c ca de base.
  \end{frame}

  \begin{frame}
    Note que que $p_T$ ter raiz, \'e uma condi\c c\~ao suficiente para que o n\'ucleo de ${\lambda-T}$ seja n\~ao trivial ($\{0\}$), pois neste caso o operador \'e n\~ao invert\'ivel. Doravante, denotaremos
    \[
      A_\lambda=\ker(\lambda-T)=\bigl\{v:v\in V\,\a\,(T-\lambda)v=0\bigr\}.
    \]
    E o referiremos como {\bf auto-espa\c co} associado ao autovalor $\lambda$.
  \end{frame}

  \begin{frame}{Vale a pena indagar}
  E se a matriz da transforma\c c\~ao n\~ao for quadrada? Neste caso, considera-se os autovalores de $T^tT$, cuja matriz \'e quadrada. Estes valores s\~ao chamados de {\bf valores singulares}.
  \end{frame}

  \begin{frame}
  Nesta apresenta\c c\~ao nos restringiremos \`as transforma\c c\~oes ou automorfismos em $V$, consequentemente as matrizes ser\~ao inevitalvemente quadradas. 
  \end{frame}

  \begin{frame}
    A seguir veremos um m\'etodo de como obter explicitamente os valores caracter\'isticos diretamente da representa\c c\~ao matricial numa determinada base.
  \end{frame}

  \section{Diagonaliza\c c\~ao de operadores}

  \begin{frame}
   A t\'itulo de ilustra\c c\~ao, consideremos $\lambda_i$ com $i\in\{1,\ldots,r\}$ os autovalores de $T$ e, suponhamos que $V$ admite uma base de autovetores, digamos $\frak a\in V^n$. Naturalmente, temos que relativa \`a esta base a matriz de $T$, fica 
  \end{frame}

  \begin{frame}
   \[
     \begin{bmatrix}
       \lambda_1 &        &           &        &           &        & \cr
                 & \ddots &           &        &           &    0   & \cr
                 &        & \lambda_1 &        &           &        & \cr 
                 &        &           & \ddots &           &        & \cr 
                 &        &           &        & \lambda_r &        & \cr 
                 &   0    &           &        &           & \ddots & \cr 
                 &        &           &        &           &        & \lambda_r \cr 

     \end{bmatrix}
   \]

   Conformemente, isto simplifica a determina\c c\~ao dos valores caracter\'isticos.
  \end{frame}

  \begin{frame}
    Relativo \`a base de autovetores o polin\^omio caracter\'istico fica fatorado em termos lineares m\^onicos
    \[
      p_T(\lambda)=\prod_{i=1}^r(\lambda-\lambda_i)^{m_i},
    \]
    em que $m_i$ s\~ao as {\bf multiplicidades alg\'ebricas} dos repectivos autovalores $\lambda_i$.
  \end{frame}

  \begin{frame}
    \begin{Def}\justifying%
      Sejam $V\neq\{0\}$ um espa\c co vetorial e $T:V\ra V$ um operador linear. Diremos que $T$ \'e {\bf diagonaliz\'avel} quando $V$ admite uma base composta de autovetores.
    \end{Def}
  \end{frame}
  
  \begin{frame}
    \begin{exe}[Operador n\~ao diagnoaliz\'avel]\justifying%
      Considere um corpo algebricamente fechado $\K$, i.e., todo polin\^omio n\~ao nulo em $\K[x]$ admite raiz. Sejam $V=\K^2$ como espa\c co vetorial sobre $\K$, e $T:V\ra V$ operador cuja matriz na base can\^onica de $V$ \'e
      \[
        [T]=
        \begin{bmatrix}
          0 & 1 \cr
          0 & 0 \cr
        \end{bmatrix}.
      \]
      Naturalmente,
      \[
        p_T(\lambda)=\lambda^2.
      \]
    \end{exe}
  \end{frame}


  \begin{frame}
    Destarte, s\'o existe um \'unico autovalor, a saber, $0$. Nota-se de 
    \[
      [T]=
      \begin{bmatrix}
        0 & 1 \cr
        0 & 0 \cr
      \end{bmatrix},
    \]
    que o posto de $T$ \'e $1$, segue do teorema n\'ucleo-imagem que
    \[
      \dim A_0=\dim V-\dim{\frak I}T=2-1=1
    \]
    Consequentemente, n\~ao pode existir uma base composta por autovetores. Logo, $T$, n\~ao \'e diagonaliz\'avel.

  \end{frame}

  \begin{frame}{Polin\^omios anuladores de matrizes}
    Seja $p\in\K[x]$, diremos que o polin\^omio $p$ anula uma matriz $A\in\K^{n\times n}$ se, e somente se, ao se substituir $x$ por $A$ na express\~ao de $p$ obtemos $p(A)=0$, sendo $0$ a matriz nula.
  \end{frame}

  
  \begin{frame}
    Em termos simples estamos fazendo a seguinte abstra\c c\~ao:

    \begin{enumerate}[label = \Roman*.]
      \item{%
        Consideremos 
        \[
          p(x)=a_0x^0+\ldots+a_nx^n;
          \label{eq2101261502}
        \]
      }
      \item{%
        Trocamos $x$ por $A$ em (\ref{eq2101261502}) obtendo
        \[
          p(A)=a_0A^0+\ldots+a_nA^n;
        \]
      }
      \item{
        Desejamos saber se este novo objeto, a saber, uma matriz, denotada por $p(A)$ \'e nula.
      }
    \end{enumerate}
  \end{frame}

  \begin{frame}{Polin\^omio minimal}
    \begin{Def}\justifying%
      O polin\^omio minimal de um operador $T$ \'e um polin\^omio $m_T$ possuindo as seguintes propriedades:
      \begin{enumerate}[label = \Roman*.]
        \item{%
          $m_T([T])=0$;
        }
        \item{%
          Se $p([T])=0$, ent\~ao $m_T|p$.
        }
      \end{enumerate}

    \end{Def}
  \end{frame}

  \begin{frame}{}
    \begin{teo}\justifying%
      Sejam $T$ um operador e $p_T$ seu polin\^omio caracter\'istico. Ent\~ao $p_T([T])=0$.
    \end{teo}

    Deste teorema inferimos que $m_T|p_T$, sendo $m_T$ o polin\^omio minimal de $T$. 
  \end{frame}


  \begin{frame}{Crit\'erio de diagonaliza\c c\~ao dum operador}
    \begin{teo}\justifying%
      Seja $T$ um operador. Ent\~ao $T$ \'e diagonaliz\'avel se, e somente se, o polin\^omio minimal for
      \[
        m_T(\lambda)=\prod_{i=1}^r(\lambda-\lambda_i).
      \]
      Em que os $\lambda_i$ s\~ao precisamente os autovalores distintos de $T$, ou as ra\'izes de $p_T$, o polin\^omio caracter\'istico de $T$.
    \end{teo}
  \end{frame}

  \begin{frame}{Forma de Jordan}
  \end{frame}

  \section{Aplica\c c\~oes}
  \begin{frame}{Classifica\c c\~ao das c\^onicas}
  \end{frame}

  \begin{frame}{Sistemas de equa\c c\~oes diferenciais lineares}
  \end{frame}


  \begin{frame}[noframenumbering, plain]{Bibliografia}
    \begin{thebibliography}{}
      \bibitem{Boldrini}
      BOLDRINI, JOS\'E LUIZ {\it et al\/}. {\itshape \'Algebra Linear}. $3^{\text{a}}$ ed. S\~ao Paulo: Harper \& Row do Brasil, 1980.
      \bibitem{HalmosFDVS}
      HALMOS, PAUL R. {\it Finite Dimensional Vector Spaces}. $2^\text{nd}$ ed. Mineloa, New York: Dover Publications, 2017.
      \bibitem{HoffmanKunze}
      HOFFMAN, KENNETH; KUNZE, RAY. {\itshape Linear Algebra}. $2^\text{nd}$ ed. Englewood Cliffs, New Jersey: Prentice-Hall, 1961.
    \end{thebibliography}
  \end{frame}

\end{document}

01-limites-e-continuidade-de-funcoes-reais-de-uma-variavel
02-definicao-de-derivada-propriedades-reta-tangente-e-exemplos
03-aplicacoes-de-derivada-maximos-e-minimos-locais-e-absolutos-graficos-de-funcoes
04-teorema-do-valor-medio-e-aplicacoes
05-vetores-produto-interno-produto-vetorial-e-produto-misto-de-vetores
06-retas-e-planos-no-espaco
07-conicas-circunferencia-elipse-parabola-e-hiperbole
08-espacos-vetoriais-subespacos-subespacos-gerados-e-base
09-transformacoes-lineares
10-autovalores-autovetores-diagonalizacao
