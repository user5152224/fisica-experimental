%!latex
\documentclass[
  leqno,
  aspectratio=169
]{beamer}

% O pacote amsmath é carregado automaticamente pela classe BEAMER
\usepackage{
  amsmath,
  dsfont,
  enumitem,
  helvet,
  hyperref,
  lipsum,
  mathrsfs,
  ragged2e,
  tikz,
  xcolor,
}

\usepackage[brazil]{babel}

\usetikzlibrary{
  arrows.meta,
  calc,
}

% Controlar o estilo do Beamer   
\beamertemplatenavigationsymbolsempty 
\newcommand*{\theorembreak}{\usebeamertemplate{theorem end}\framebreak\usebeamertemplate{theorem begin}}
\setbeamercolor{section number projected}{bg=structure,fg=white}
\setbeamercolor{bibliography entry author}{fg=black}
\setbeamercolor{structure}{fg=green!50!black}
\setbeamerfont{frametitle}{shape=\slshape, family=\sffamily}  
%\setbeamerfont{section number projected}{size=\normalsize}
%\setbeamerfont{title}{shape=\itshape, family=\rmfamily}
\setbeamertemplate{bibliography item}{\insertbiblabel}
\setbeamertemplate{section in toc}[sections numbered]
\setbeamerfont{section in toc}{shape=\slshape, family=\sffamily}
\setbeamertemplate{theorems}[numbered]
\apptocmd{\frame}{}{\justifying}{} %Justificar todo o texto no frame

\newtheorem{axi}{Axioma}
\newtheorem{cor}{Corol\'ario}
\newtheorem{Def}{Defini\c c\~ao}
\newtheorem{exe}{Exemplo}
\newtheorem{lem}{Lema}
\newtheorem{obs}{Observa\c c\~ao}
\newtheorem{pro}{Proposi\c c\~ao}
\newtheorem{teo}{Teorema}

% Fazer o título da apresentação.
\newcommand{\titulo}[3]{
  \begingroup    
    \setbeamertemplate{headline}{}
    \begin{frame}[noframenumbering]    
      \centering     
      \footnotesize
      \parbox[c]{6cm}{  
        {% 
          UNIVERSIDADE\hfill FEDERAL\hfill DA\hfill PARA\'IBA%
        }
        \vskip 1pt
        \fontsize{7}{7}\selectfont
        CENTRO \hfill DE\hfill CI\^ENCIAS\hfill EXATAS\hfill E\hfill DA NATUREZA%
        \par
      }      
      \vskip 1.4cm
      {
        \usebeamercolor[fg]{structure}
        \large\bfseries\sffamily\slshape\MakeUppercase{%
          #1
        }
      }
      \vskip .42cm
      {%
        \itshape por
      }
      \vskip .42cm
      {%
        \sffamily #2
      }
      \vskip .2cm
      e
      {%
        \sffamily #3
      }
      \tiny\par
      {%
        \tt\href{mailto:harllen.araujo@gmail.com}{<harllen.araujo@gmail.com>}
      }
      \vskip .63cm
      Jo\~ao Pessoa\par\the\year
    \end{frame}
  \endgroup
}

% Barra de progresso adaptado de: https://tex.stackexchange.com/questions/59742/progress-bar-for-latex-beamer.
\makeatletter
  \def\progressbar@progressbar{}  % the progress bar
  \newcount\progressbar@tmpcounta % auxiliary counter
  \newcount\progressbar@tmpcountb % auxiliary counter
  \newdimen\progressbar@pbht      % progressbar height
  \newdimen\progressbar@pbwd      % progressbar width
  \newdimen\progressbar@rcircle   % radius for the circle
  \newdimen\progressbar@tmpdim    % auxiliary dimension
  
  \progressbar@pbwd=\linewidth
  \progressbar@pbht=1pt
  
  % A barra de progresso
  \def\progressbar@progressbar{%
    \progressbar@tmpcounta=\insertframenumber
    \progressbar@tmpcountb=\inserttotalframenumber
       \progressbar@tmpdim=\progressbar@pbwd
    \multiply\progressbar@tmpdim by \progressbar@tmpcounta
    \divide\progressbar@tmpdim by \progressbar@tmpcountb
  
    \begin{tikzpicture}
      \draw [ white!85!black,
              line width=\progressbar@pbht,
              line cap=round,
            ] (0pt, 0pt) -- ++ (\progressbar@pbwd,0pt) node [ text=green!50!black, 
                                                              midway,
                                                              below=2pt
                                                            ] {
                                                                %\fontsize{9.1}{9.1}\selectfont
                                                                \slshape\bfseries
                                                                \MakeUppercase{\@currentlabelname}
                                                              };
      \draw [ green!50!black,
              line width=\progressbar@pbht,
              line cap=round,
            ] (0pt, 0pt) -- ++ (\the\dimexpr\progressbar@tmpdim-\progressbar@rcircle\relax,0pt);
      \node[ %draw=green!50!black,
             text width=3.5em,
             align=center,
             inner sep=1pt,
             text=green!50!black,
             anchor=west,
             rounded corners=.5pt
           ] at (\progressbar@pbwd,0) {%
                                        \fontsize{7.7}{7.7}\selectfont\insertframenumber/\inserttotalframenumber
                                      };
    \end{tikzpicture}%
  }
  \addtobeamertemplate{headline}{}
  {%
    \begin{beamercolorbox}[ wd=\paperwidth,
                            ht=7ex,
                            center,
                            dp=1ex
                          ]{white}%
      \progressbar@progressbar%
    \end{beamercolorbox}%
  }
\makeatother

% Define um estilo de teorema
\newtheoremstyle{slanted}
{10pt}
{10pt}
{\slshape}
{}
{\bfseries}
{}
{0.5em}
{\thmname{\bfseries #1} \thmnumber{\scshape #2}\thmnote{ \scshape(#3)}\quad}

\theoremstyle{slanted}

% Referências cruzadas.
\def\rf#1{
  {
    \usebeamercolor[fg]{structure}\ff\ref{#1}
  }
}

\def\prova{
  {
    \usebeamercolor[fg]{structure}\itshape Prova.\ %
  }
}
\def\AEs{{\bf AEs}}
\def\AEx{{\bf AEx}}
\def\AS{{\bf AS}\ }
\def\Cap{\mathop{\textstyle\bigcap}}
\def\Cup{\mathop{\textstyle\bigcup}}
\def\C{\mathds{C}}
\def\Join{\mathop{\textstyle\bigcurlyvee}}
\def\K{\mathds{K}}
\def\L{\mathds{L}}
\def\N{\mathds{N}}
\def\Meet{\mathop{\textstyle\bigcurlywedge}}
\def\Q{\mathds{Q}}
\def\R{\mathds{R}}
\def\U{{\cal U}}
\def\Vee{\mathop{\textstyle\bigvee}}
\def\Wedge{\mathop{\textstyle\bigwedge}}
\def\Z{\mathds{Z}}
\def\[{\begin{equation}}
\def\]{\end{equation}}
\def\af{{\it a fortiori}}
\def\a{\wedge}
\def\bf{\bfseries}
\def\cal{\mathcal}
\def\cp{\mathbin{\neg}}
\def\ct#1{{[#1 \citenum{#1}]}}
\def\d{\,d}
\def\fr#1#2{\frac{#1}{#2}}
\def\frak{\mathfrak}
\def\join{\curlyvee}
\def\la{\leftarrow}
\def\lge{\langle}
\def\lla{\longleftarrow}
\def\llra{\longleftrightarrow}
\def\lra{\longrightarrow}
\def\meet{\curlywedge}
\def\ora#1{\mkern3mu\overrightarrow{\mkern-3mu#1\mkern3mu}\mkern-3mu}
\def\o{\vee}
\def\pma{{\bfseries PMA}}
\def\prova{\noindent{\usebeamercolor[fg]{structure}\bfseries Prova}\quad}
\def\qed{\nopagebreak\hbox{ }\hfill{\usebeamercolor[fg]{structure}\rule{.42em}{1em}}\bigskip}
\def\ra{\rightarrow}
\def\res{\noindent{\itshape\bfseries Resolu\c c\~ao.\/}\ }
\def\rf#1#2{{\usebeamercolor[fg]{structure}\bfseries\hyperref[#2]{#1 \ref{#2}}}}
\def\rge{\rangle}
\def\scr#1{\mathscr{#1}}
\def\sen{\,{\rm sen}\,}
\def\tvm{{\bf TVM}}
\def\prll{\mathbin{\hbox{/\kern-1pt/}}}
\def\olra#1{\overleftrightarrow{\mkern-4mu#1}}



\begin{document}
  \titulo{%
    {\footnotesize F\'isica Experimental I:}
    \vskip 3pt
    Movimento Retil\'ineo Uniforme
  }{
    Harllen Ara\'ujo de Sena
  }{
    Jos\'e Willian Gon\c calves da Silva 
  }

  \begin{frame}[noframenumbering, plain]{Sum\'ario}
      \tableofcontents[pausesections]
  \end{frame}
  
  \section{Objetivo}
  \begin{frame}{Objetivo}\justifying
    O objetivo principal do experimento foi estudar o movimento uniforme. A ideia principal era atingir condi\c c\~oes que se aproximassem do ideal, i.e., velocidade constante.
  \end{frame}

  \begin{frame}{Qual a pergunta a ser respondida?}
    Sob condi\c c\~oes restritas, com aproxima\c c\~oes grosseiras, podemos validar o movimento uniforme? Em outros termos, queremos responder razovelmente se o princ\'ipio da in\'ercia de Galileo (Galileu):

    \vskip 1cm
    
    \hfil
    \parbox[c]{10cm}{\sl%
      ``Um corpo tende a manter velocidade constante, a menos que uma acelera\c c\~ao $($for\c ca\/$)$ seja aplicada sobre ele.''
    }
    \hfil
    \vskip 1cm
    \'e v\'alido.
  \end{frame}

  \section{Fundamente\c c\~ao te\'orica}
  \begin{frame}{Equa\c c\~ao principal}
    Dado um sistema de cordenadas tridimensional, a equa\c c\~ao hor\'aria para o movemento uniforme \'e dada por uma fun\c c\~ao $\ora{x}:[0,\infty[\,\ra \R^3$ definida por
    \[
      x(t)=\ora{x_0}+\ora{v}t.
    \]
    com $\ora{x_0}\in\R^3$.
  \end{frame}

  \begin{frame}{Equa\c c\~ao principal}
    Como o movimento \'e unidimensional podemos escrever simplesmente $x:[0,\infty[\,\ra \R$ definida por
    \[
      x(t)=x_0+vt.
    \]
    com $x_0\in\R$.
  \end{frame}

  \begin{frame}{O que esperamos observar?}
    Simplesmente, velocidade aproximadamente constante. Neste caso, como se sabe a velocidade m\'edia entre dois quaisquer instantes da trajet\'oria \'e igual a instant\^anea. 
  \end{frame}

  \begin{frame}{O que esperamos observar?}
    Destarte, medindo-se velocidades m\'edias e as comparando, devemos observar que elas n\~ao devem divergir muito.
  \end{frame}
  
  \section{Metodologia}
  \begin{frame}{Instrumentos utilizados}
    \begin{enumerate}[label = \Roman*.]
      \item{%
        R\'egua;
      }
      \item{%
        Paqu\'imetro;
      }
      \item{%
        Fotogate;
      }
      \item{%
        Trilho pneum\'atico.
      }
    \end{enumerate}
  \end{frame}

  \begin{frame}{Como os dados foram coletados?}
    Usamos um trilho pneum\'atico de 2m de comprimento. T\'inhamos a nossa disposi\c c\~ao quatro sensores \'oticos do fotogate, estes foram espa\c cados uniformemente uns dos outros a uma dist\^ancia de aproximadamente 40cm.
  \end{frame}
  
  \begin{frame}{Quais as grandezas que foram medidas diretamente?}
    Precipuamente (essencialmente), foram o {\bf comprimento} em metro (m) e seus sub\-m\'ultiplos por meio de r\'egua e paqu\'imetro, e o {\bf tempo} em segundos (s) e seus subm\'ultiplos via fotogate.
  \end{frame}

  \section{Dados experimentais}

  \begin{frame}{Tabela $($r\'egua$)$}

    {
      \centering
      \begin{table}
        \begin{tabular}{c c c c c c}
            TM (s) & DP (s) & ET (s) & VM (cm/s) & PE (cm/s) & II (cm/s)  \cr
            0.26 & 0.004 & 0.001 & 38.0 &  0.3 & $[37.7,38.2]$ \cr
            0.27 & 0.004 & 0.001 & 37.4 &  0.2 & $[37.2,37.7]$ \cr
            0.27 & 0.004 & 0.001 & 36.8 &  0.3 & $[36.5,37.0]$ \cr
            0.27 & 0.004 & 0.001 & 36.6 &  0.2 & $[36.4,36.9]$ \cr
        \end{tabular}
        \caption{tabela segundo a r\'egua.}
      \end{table}
    }
    {\scriptsize%
      {\bf Legenda:}

      \medskip
      \begin{tabular}{rl}
        TM: & Tempo m\'edio;\cr
        DP: & Desvio padr\~ao;\cr
        ET: & Erro total;\cr
        VM: & Velocidade m\'edia;\cr
        II: & Intervalo de incerteza.\cr
      \end{tabular}
    }
  \end{frame}

  \begin{frame}{Tabela $($paqu\'imetro$)$}

    {
      \centering
      \begin{table}
        \begin{tabular}{c c c c c c}
          TM (s) & DP (s) & ET (s) & VM (cm/s) & PE (cm/s) & II (cm/s)  \cr
          0.26 & 0.004 & 0.001 & 37.0 & 0.2 & [36.8,37.2] \cr
          0.27 & 0.004 & 0.001 & 36.5 & 0.2 & [36.3,36.7] \cr
          0.27 & 0.004 & 0.001 & 35.9 & 0.2 & [35.7,36.0] \cr
          0.27 & 0.004 & 0.001 & 35.7 & 0.2 & [35.6,35.9] \cr
        \end{tabular}
        \caption{tabela segundo a r\'egua.}
      \end{table}
    }
    {\scriptsize%
      {\bf Legenda:}

      \medskip
      \begin{tabular}{rl}
        TM: & Tempo m\'edio;\cr
        DP: & Desvio padr\~ao;\cr
        ET: & Erro total;\cr
        VM: & Velocidade m\'edia;\cr
        II: & Intervalo de incerteza.\cr
      \end{tabular}
    }
  \end{frame}


  \begin{frame}{Gr\'afico $($r\'egua$)$}
    \centering
    \vfill
    \begin{figure}
      \tikzpicture[scale=7]
       \scope[{Circle}-{Circle}]
         \draw (37.7, .3) node [above=2pt] {37.7} -- (38.2, .3) node [above=2pt] {38.2}; 
         \draw (37.2, .2) node [above=2pt] {37.2} -- (37.7, .2) node [above=2pt] {37.7};
         \draw (36.5, .1) node [above=2pt] {36.5} -- (37.0, .1) node [above=2pt] {37.0};
         \draw (36.4, .0) node [above=2pt] {36.4} -- (36.9, .0) node [above=2pt] {36.9};
       \endscope
      \endtikzpicture
      \caption{medi\c c\~oes segundo a r\'egua.}
    \end{figure}
    \vfill
  \end{frame}

  \begin{frame}{Gr\'afico $($paqu\'imetro$)$}
    \centering
    \vfill
    \begin{figure}
      \tikzpicture[scale=7]
        \scope[{Circle}-{Circle}]
          \draw (36.8, .3)  node [above=2pt] {36.8} -- (37.2, .3) node [above=2pt] {37.2} ; 
          \draw (36.3, .2)  node [above=2pt] {36.3} -- (36.7, .2) node [above=2pt] {36.7} ;
          \draw (35.7, .1)  node [above=2pt] {35.7} -- (36.0, .1) node [above=2pt] {36.0} ;
          \draw (35.6, .0)  node [above=2pt] {35.6} -- (35.9, .0) node [above=2pt] {35.9} ;
        \endscope
      \endtikzpicture
      \caption{Medi\c c\~oes segundo o paqu\'imetro.}
    \end{figure}
    \vfill
  \end{frame}

  \section{An\'alise}
  \begin{frame}{C\'alculos realizados}
    \[
      \overline{t} = {\sum_{i=1}^nt_i\over n}\qquad(\hbox{tempo m\'edio, m\'edia aritm\'etica})
    \]
    \[
      \sigma = \sqrt{{\sum (\overline{t}-t_i)^2\over n-1}}\qquad(\hbox{desvio padr\~ao})
    \]
    \[
      \sigma_{\overline{t}}={\sigma\over\sqrt{n}}\qquad(\hbox{erro estat\'istico})
    \]
  \end{frame}

  \begin{frame}{C\'alculos realizados}

    \[
      \delta_T = \sqrt{\Bigl({\partial v\over\partial t}\cdot\overline{t}\Bigr)^2+\Bigr({\partial v\over\partial d}\cdot\varepsilon_{\rm inst}\Bigr)^2}\qquad(\hbox{propaga\c c\~ao de erros}),
    \]
    em que $\varepsilon_{\rm inst}$ \'e o a incerteza instrumental (erro instrumental) e 
    \[
      v(d,t)={d\over t}
    \]
    \'e a fun\c c\~ao velocidade.
  \end{frame}

  \begin{frame}{Sum\'ario dos c\'alculos $($paqu\'imetro$)$}
    \centering                               
    \begin{table}                            
      \begin{alignat*}{2}                    
        37.0 & \pm 0.2\, \hbox{\sffamily cm/s}\cr
        36.5 & \pm 0.2\, \hbox{\sffamily cm/s}\cr
        35.9 & \pm 0.2\, \hbox{\sffamily cm/s}\cr
        35.7 & \pm 0.2\, \hbox{\sffamily cm/s}\cr
      \end{alignat*}
      \caption{velocidade m\'edia com suas respectivas incertezas.}
    \end{table}
  \end{frame}

  \section{Discuss\~ao}
  \begin{frame}{O resultado tem sentido f\'isico?}
    Sim, apesar de n\~ao termos atingido a velocidade aproximadamente constante em nossa experimenta\c c\~ao, alguns fatores f\'isicos entram em jogo.
  \end{frame}

  \begin{frame}{O que entra em jogo?}

    {\bf Atrito.} quanto mais intensa a vaz\~ao de ar para o trilho pneum\'atico tanto melhor ser\'a o colch\~ao de ar, tanto menor ser\'a o atrito, melhorando os resultados.
  \end{frame}

  \begin{frame}{Fontes de erro}

    {\bf Contagem do fotogate.} Quanto maior a enegia potencial el\'astica, tanto maior ser\'a a energia cin\'etica, em consequ\^encia ser\'a maior a velocidade. Como o comprimento da bandeira \'e curto, mais pr\'oximo de zero ser\'a o tempo.
    \[
      t={d\over v}.
    \]
  \end{frame}
  \begin{frame}{Fontes de erro}

    Aqui o fotogate introduz imprecis\~ao ao experimento, pois ele n\~ao conta as casas decimais necess\'arias. O que segue \'e s\'o um exemplo grosseiro:
    $$
      \vbox{
        \hbox{0.018\color{red}{314}}
        \hbox{0.017\color{red}{571}}
      }
    $$
  \end{frame}



  %\begin{frame}[noframenumbering, plain]{Bibliografia}
  %  \begin{thebibliography}{}
  %  \end{thebibliography}
  %\end{frame}

\end{document}
