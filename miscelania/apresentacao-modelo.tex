% O pacote amsmath é carregado automaticamente pela classe BEAMER
\usepackage{
  amsmath,
  dsfont,
  enumitem,
  helvet,
  hyperref,
  lipsum,
  mathrsfs,
  ragged2e,
  tikz,
  xcolor,
}

\usepackage[brazil]{babel}

\usetikzlibrary{
  arrows.meta,
  calc,
}

% Controlar o estilo do Beamer   
\beamertemplatenavigationsymbolsempty 
\newcommand*{\theorembreak}{\usebeamertemplate{theorem end}\framebreak\usebeamertemplate{theorem begin}}
\setbeamercolor{section number projected}{bg=structure,fg=white}
\setbeamercolor{bibliography entry author}{fg=black}
\setbeamercolor{structure}{fg=green!50!black}
\setbeamerfont{frametitle}{shape=\slshape, family=\sffamily}  
%\setbeamerfont{section number projected}{size=\normalsize}
%\setbeamerfont{title}{shape=\itshape, family=\rmfamily}
\setbeamertemplate{bibliography item}{\insertbiblabel}
\setbeamertemplate{section in toc}[sections numbered]
\setbeamerfont{section in toc}{shape=\slshape, family=\sffamily}
\setbeamertemplate{theorems}[numbered]
\apptocmd{\frame}{}{\justifying}{} %Justificar todo o texto no frame

\newtheorem{axi}{Axioma}
\newtheorem{cor}{Corol\'ario}
\newtheorem{Def}{Defini\c c\~ao}
\newtheorem{exe}{Exemplo}
\newtheorem{lem}{Lema}
\newtheorem{obs}{Observa\c c\~ao}
\newtheorem{pro}{Proposi\c c\~ao}
\newtheorem{teo}{Teorema}

% Fazer o título da apresentação.
\newcommand{\titulo}[3]{
  \begingroup    
    \setbeamertemplate{headline}{}
    \begin{frame}[noframenumbering]    
      \centering     
      \footnotesize
      \parbox[c]{6cm}{  
        {% 
          UNIVERSIDADE\hfill FEDERAL\hfill DA\hfill PARA\'IBA%
        }
        \vskip 1pt
        \fontsize{7}{7}\selectfont
        CENTRO \hfill DE\hfill CI\^ENCIAS\hfill EXATAS\hfill E\hfill DA NATUREZA%
        \par
      }      
      \vskip 1.4cm
      {
        \usebeamercolor[fg]{structure}
        \large\bfseries\sffamily\slshape\MakeUppercase{%
          #1
        }
      }
      \vskip .42cm
      {%
        \itshape por
      }
      \vskip .42cm
      {%
        \sffamily #2
      }
      \vskip .2cm
      e
      {%
        \sffamily #3
      }
      \tiny\par
      {%
        \tt\href{mailto:harllen.araujo@gmail.com}{<harllen.araujo@gmail.com>}
      }
      \vskip .63cm
      Jo\~ao Pessoa\par\the\year
    \end{frame}
  \endgroup
}

% Barra de progresso adaptado de: https://tex.stackexchange.com/questions/59742/progress-bar-for-latex-beamer.
\makeatletter
  \def\progressbar@progressbar{}  % the progress bar
  \newcount\progressbar@tmpcounta % auxiliary counter
  \newcount\progressbar@tmpcountb % auxiliary counter
  \newdimen\progressbar@pbht      % progressbar height
  \newdimen\progressbar@pbwd      % progressbar width
  \newdimen\progressbar@rcircle   % radius for the circle
  \newdimen\progressbar@tmpdim    % auxiliary dimension
  
  \progressbar@pbwd=\linewidth
  \progressbar@pbht=1pt
  
  % A barra de progresso
  \def\progressbar@progressbar{%
    \progressbar@tmpcounta=\insertframenumber
    \progressbar@tmpcountb=\inserttotalframenumber
       \progressbar@tmpdim=\progressbar@pbwd
    \multiply\progressbar@tmpdim by \progressbar@tmpcounta
    \divide\progressbar@tmpdim by \progressbar@tmpcountb
  
    \begin{tikzpicture}
      \draw [ white!85!black,
              line width=\progressbar@pbht,
              line cap=round,
            ] (0pt, 0pt) -- ++ (\progressbar@pbwd,0pt) node [ text=green!50!black, 
                                                              midway,
                                                              below=2pt
                                                            ] {
                                                                %\fontsize{9.1}{9.1}\selectfont
                                                                \slshape\bfseries
                                                                \MakeUppercase{\@currentlabelname}
                                                              };
      \draw [ green!50!black,
              line width=\progressbar@pbht,
              line cap=round,
            ] (0pt, 0pt) -- ++ (\the\dimexpr\progressbar@tmpdim-\progressbar@rcircle\relax,0pt);
      \node[ %draw=green!50!black,
             text width=3.5em,
             align=center,
             inner sep=1pt,
             text=green!50!black,
             anchor=west,
             rounded corners=.5pt
           ] at (\progressbar@pbwd,0) {%
                                        \fontsize{7.7}{7.7}\selectfont\insertframenumber/\inserttotalframenumber
                                      };
    \end{tikzpicture}%
  }
  \addtobeamertemplate{headline}{}
  {%
    \begin{beamercolorbox}[ wd=\paperwidth,
                            ht=7ex,
                            center,
                            dp=1ex
                          ]{white}%
      \progressbar@progressbar%
    \end{beamercolorbox}%
  }
\makeatother

% Define um estilo de teorema
\newtheoremstyle{slanted}
{10pt}
{10pt}
{\slshape}
{}
{\bfseries}
{}
{0.5em}
{\thmname{\bfseries #1} \thmnumber{\scshape #2}\thmnote{ \scshape(#3)}\quad}

\theoremstyle{slanted}

% Referências cruzadas.
\def\rf#1{
  {
    \usebeamercolor[fg]{structure}\ff\ref{#1}
  }
}

\def\prova{
  {
    \usebeamercolor[fg]{structure}\itshape Prova.\ %
  }
}
\def\AEs{{\bf AEs}}
\def\AEx{{\bf AEx}}
\def\AS{{\bf AS}\ }
\def\Cap{\mathop{\textstyle\bigcap}}
\def\Cup{\mathop{\textstyle\bigcup}}
\def\C{\mathds{C}}
\def\Join{\mathop{\textstyle\bigcurlyvee}}
\def\K{\mathds{K}}
\def\L{\mathds{L}}
\def\N{\mathds{N}}
\def\Meet{\mathop{\textstyle\bigcurlywedge}}
\def\Q{\mathds{Q}}
\def\R{\mathds{R}}
\def\U{{\cal U}}
\def\Vee{\mathop{\textstyle\bigvee}}
\def\Wedge{\mathop{\textstyle\bigwedge}}
\def\Z{\mathds{Z}}
\def\[{\begin{equation}}
\def\]{\end{equation}}
\def\af{{\it a fortiori}}
\def\a{\wedge}
\def\bf{\bfseries}
\def\cal{\mathcal}
\def\cp{\mathbin{\neg}}
\def\ct#1{{[#1 \citenum{#1}]}}
\def\d{\,d}
\def\fr#1#2{\frac{#1}{#2}}
\def\frak{\mathfrak}
\def\join{\curlyvee}
\def\la{\leftarrow}
\def\lge{\langle}
\def\lla{\longleftarrow}
\def\llra{\longleftrightarrow}
\def\lra{\longrightarrow}
\def\meet{\curlywedge}
\def\ora#1{\mkern3mu\overrightarrow{\mkern-3mu#1\mkern3mu}\mkern-3mu}
\def\o{\vee}
\def\pma{{\bfseries PMA}}
\def\prova{\noindent{\usebeamercolor[fg]{structure}\bfseries Prova}\quad}
\def\qed{\nopagebreak\hbox{ }\hfill{\usebeamercolor[fg]{structure}\rule{.42em}{1em}}\bigskip}
\def\ra{\rightarrow}
\def\res{\noindent{\itshape\bfseries Resolu\c c\~ao.\/}\ }
\def\rf#1#2{{\usebeamercolor[fg]{structure}\bfseries\hyperref[#2]{#1 \ref{#2}}}}
\def\rge{\rangle}
\def\scr#1{\mathscr{#1}}
\def\sen{\,{\rm sen}\,}
\def\tvm{{\bf TVM}}
\def\prll{\mathbin{\hbox{/\kern-1pt/}}}
\def\olra#1{\overleftrightarrow{\mkern-4mu#1}}

